% coding:utf-8

%----------------------------------------
%FOSAEBV, a LaTeX-Code for a summary of realtime image processing
%Copyright (C) 2013, Mario Felder

%This program is free software; you can redistribute it and/or
%modify it under the terms of the GNU General Public License
%as published by the Free Software Foundation; either version 2
%of the License, or (at your option) any later version.

%This program is distributed in the hope that it will be useful,
%but WITHOUT ANY WARRANTY; without even the implied warranty of
%MERCHANTABILITY or FITNESS FOR A PARTICULAR PURPOSE.  See the
%GNU General Public License for more details.
%----------------------------------------

\chapter{Fourier-Transformation}

\section{1D Fourier-Transformation}
Die Fourier Transformationen dient dazu, für eine Funktion $h(x)$ im Ortsraum das zugehörige Frequenzspektrum im Ortsfrequenzraum $\hat{h}(f)$  zu bestimmen.
Dabei gibt es grundsätzlich vier mögliche Anwendungsfälle und zugehörige Formulierungen der Fourier Transformation (FT):

\paragraph{Kontinuierliche FT einer aperiodischen Funktion $h(x)$:}
\[
	\hat{h}(f) = \int_{-\infty}^{\infty} h(x) \cdot \e^{-\im \cdot 2\pi \cdot f \cdot x}\di x \qquad
	h(f) = \int_{-\infty}^{\infty} \hat{h}(x) \cdot \e^{\im \cdot 2\pi \cdot f \cdot x}\di f
\]

\paragraph{Kontinuierliche FT einer $X_0$-periodischen Funktion $h(x)$:}
\[
	\hat{h}[n] = \frac{1}{X_0}\int_{0}^{X_0} h(x) \cdot \e^{-\im \cdot 2\pi \cdot f_0 \cdot x}\di x \qquad
	h(f) = \sum_{n=-\infty}^{\infty} \hat{h}[n] \cdot \e^{\im \cdot 2\pi \cdot n \cdot f_0 \cdot x}
\]
\begin{footnotesize}
	$f_0 = \frac{1}{X_0}$\\
\end{footnotesize}
~\\
Im Falle einer periodischen Funktion besteht das Frequenzspektrum nur aus diskreten Werten an den Frequenzen $f_n = n \cdot f_0$.

\paragraph{Diskrete FT einer aperiodischen Funktion für äquidistante Abtastpunkte $h_n = h(n\cdot x_s)$:}
\[
	\hat{h}(f) = \sum_{n = -\infty}^{\infty} h[n] \cdot \e^{-\im \cdot 2\pi \cdot f \cdot n \cdot x_s} \qquad
	h[n] = \frac{1}{f_s}\int_{0}^{f_s} \hat{h}(f) \cdot \e^{\im \cdot 2\pi \cdot n \cdot x_s \cdot f}\di f
\]
\begin{footnotesize}
	$f_s = \frac{1}{x_s}$\\
\end{footnotesize}
~\\
Bei Abtastung eines Signals mit dem Intervall $x_s$ entsteht ein periodisches Frequenzspektrum der Periode $f_s$.

\paragraph{Diskrete FT einer $X_0$-periodischer Funktion für äquidistante Abtastpunkte $h_n = h(n\cdot x_s)$:}
\[
	\hat{h}[k] = \sum_{n=0}^{N-1} h[n] \cdot \e^{-\im \cdot 2\pi \cdot \frac{k \cdot n}{N}} \qquad
	h[n] = \frac{1}{N} \sum_{k=0}^{N-1} \hat{h}[k] \cdot \e^{\im \cdot 2\pi \cdot \frac{k \cdot n}{N}}
\]
\begin{footnotesize}
	$x_s =\frac{X_0}{N}$\\
\end{footnotesize}
~\\
Wird ein periodisches Signal mit $n \cdot x_0$ Intervallpunkten abgetastet, so hat die FT ein ebenfalls diskretes und periodisches Frequenzspektrum der Periode $f_s = \frac{1}{x_s} = \frac{N}{X_0}$.
\\

\paragraph{Rechenregeln\\}
\begin{tabular}{ll}
	Linearität: & $\mathcal{F}\{\lambda \cdot h(x)\} = \lambda \cdot \mathcal{F}\{h(x)\}$ \\
	 & $\mathcal{F}\{h(x) + g(x)\} = \mathcal{F}\{h(x)\} + \mathcal{F}\{g(x)\}$\\
	Verschiebung: & $\mathcal{F}\{h(x+x_0)\} = \e^{\im \cdot 2\pi \cdot f \cdot x_0} \cdot \mathcal{F}\{h(x)\}$\\
	Faltungssatz: & $\mathcal{F}\{h(x) \otimes g(x)\} = \mathcal{F}\{h(x)\} \cdot \mathcal{F}\{g(x)\}$\\
\end{tabular}

\section{Diskrete 2D Fourier-Transformation}
Bei Annahme dass die Funktion in $x$-Richtung $X_0$-periodisch und in $y$-Richtung $Y_0$-periodisch ist ($X_0 = N \cdot x_s$, $Y_0 = M \cdot y_s$), so ist die 2D DFT folgendermassen definiert:
\[\begin{aligned}
	\hat{h}[l,k] &= \sum_{m=0}^{M-1} \sum_{n=0}^{N-1} h[m,n] \cdot \e^{-\im \cdot 2\pi \cdot \left[\frac{l \cdot m}{M} + \frac{k \cdot n}{N}\right]}
	\\
	h[m,n] &= \frac{1}{M \cdot N} \sum_{l=0}^{M-1} \sum_{k=0}^{N-1} \hat{h}[l,k] \cdot \e^{\im \cdot 2\pi \cdot \left[\frac{l \cdot m}{M} + \frac{k \cdot n}{N}\right]}
\end{aligned}\]
~\\\\
Lösung in MATLAB:
\lstset{language=Matlab}
\lstinputlisting[firstline=1,caption=]{../Matlab/dft.m}
~\\

\section{Periodische Strukturen}
Periodische Strukturen führen auch zu einem periodischen Frequenzspektrum.
Wenn das Bild jedoch nicht über den Bildrand hinaus periodisch ist, treten Fehler in der DFT auf.
Dies kann korrigiert werden, indem eine Fensterfunktion angewendet wird, welche zum Rand des Bildes stetig nach Null abfällt.
~\\\\
Lösung in MATLAB:
\lstset{language=Matlab}
\lstinputlisting[firstline=1,caption=]{../Matlab/hann.m}
~\\

\section{Unterdrückung periodischer Störungen}
Periodische Störungen haben auch eine periodische Auswirkung im Frequenzspektrum.
Durch einen Nochtfilter können diese selektiv ausgeblendet werden. So kann ein Grossteil des Rauschens eliminiert werden.
~\\\\
Lösung in MATLAB:
\lstset{language=Matlab}
\lstinputlisting[firstline=1,caption=]{../Matlab/notch.m}
~\\

\section{Deconvolution}
Ein Bild, welches durch einen Filter unscharf gezeichnet wurde, kann durch Deconvolution wieder hergestellt werden. Dabei wird der Faltungssatz angewendet.
~\\\\
Lösung in MATLAB:
\lstset{language=Matlab}
\lstinputlisting[firstline=1,caption=]{../Matlab/deconvolution.m}
~\\

% coding:utf-8

%----------------------------------------
%FOSAEBV, a LaTeX-Code for a summary of realtime image processing
%Copyright (C) 2013, Mario Felder

%This program is free software; you can redistribute it and/or
%modify it under the terms of the GNU General Public License
%as published by the Free Software Foundation; either version 2
%of the License, or (at your option) any later version.

%This program is distributed in the hope that it will be useful,
%but WITHOUT ANY WARRANTY; without even the implied warranty of
%MERCHANTABILITY or FITNESS FOR A PARTICULAR PURPOSE.  See the
%GNU General Public License for more details.
%----------------------------------------

\section{Lineare Filter - Faltung}
Mathematische Beschreibung der Faltung $I \otimes h$ eines Bildes $I_{m,n}$ mit einer Maske $h_{p,q}$:
\[
	I \otimes h:I_{m,n} \rightarrow \sum_{p=-u}^{u}\sum_{q=-v}^{v} I_{m-p,n-q} \cdot h_{p,q}
\]
\paragraph{Rechengesetzte:}
\begin{itemize}
	\item[]Kommutativität: $I \otimes J = J \otimes I$
	\item[]Assoziativität: $(I \otimes J) \otimes K = I \otimes (J \otimes K)$
	\item[]Distributivität: $I \otimes (J + K) = I \otimes J + I \otimes K$
	\item[]skalare Assoziativität: $a \cdot (I \otimes J) = (a \cdot I) \otimes J = I \otimes (a \cdot J)$
\end{itemize}
~\\
Durch Anwendung der Rechenregeln kann eine Maske effizient angewendet werden,
indem sie aus zwei eindimensionalen Masken zusammengesetzt wird:
\[
	I \otimes h = (I \otimes h_x) \otimes h_y
\]

\subsection{Glätten von Bildern (Tiefpass)}
Bei der Glättung wird ein Pixel durch die Mittelung des Pixels mit den Nachbarpixeln ersetzt.
Dies dient zur Rauschunterdrückung oder als Vorstufe zum Dezimieruren der räumlichen Auflösung (eng. down sampling).\\
Verwendete Standardfilter sind Rechteck- oder Gaussmasken:
\[
	R=\frac{1}{9} \left[\begin{matrix}
					1 & 1 & 1\\
					1 & 1 & 1\\
					1 & 1 & 1\\	\end{matrix}\right] \qquad
	G=\frac{1}{16} \left[\begin{matrix}
					1 & 2 & 1\\
					2 & 4 & 2\\
					1 & 2 & 1\\	\end{matrix}\right]
\]
~\\
Damit der Wertebereich der Abbildung $G' =[0\ 255]$ bleibt, muss die Summe über alle Maskenkoeffizienten $1$ ergeben.\\
\\\\
Lösung in MATLAB:
\lstset{language=Matlab}
\lstinputlisting[firstline=1,caption=]{../Matlab/Glaettung.m}
~\\

\subsection{Kantenhervorhebung (Hochpass)}
Harte Grauwertübergänge eines Bildes werden verstärkt, während weiche Übergänge abgeschwächt werden. Hierbei ist der Gradienten Filter eine wichtige Filterklasse.
Diese berechnet den Gradienten, d.h. die Ableitung der Grauwerte $I_{m,n}$ eines Bildes.
\[\begin{aligned}
	\pdifrac{I(x,y)}{x} &= \lim\limits_{\Delta x \rightarrow 0} \frac{I(x + \Delta x,y) - I(x-\Delta x,y)}{2 \cdot \Delta x} = \frac{I_{m,n+1} - I_{m,n-1}}{2}\\
	\pdifrac{I(x,y)}{y} &= \lim\limits_{\Delta y \rightarrow 0} \frac{I(x,y + \Delta y) - I(x,y-\Delta y)}{2 \cdot \Delta y} = \frac{I_{m+1,n} - I_{m-1,n}}{2}
\end{aligned}\]
~\\\\
Mit folgenden Filtern kann diese Berechnung des Gradienten durchgeführt werden:
\[
	h_x = \frac{1}{2} \cdot \left[\begin{matrix}
		-1 & 0 & 1
		\end{matrix}\right] \qquad
	h_y = \frac{1}{2} \cdot \left[\begin{matrix}
		-1 \\ 0 \\ 1
		\end{matrix}\right] \qquad
\]
~\\\\
Da diese fehleranfällig auf Rausche sind, wird jeweils mit einem Glättungsfilter kombiniert.\\
\\
Prewitt-Filter:
\begin{scriptsize}\[
	h_x = \left[\begin{matrix} 1 \\ 1 \\ 1\end{matrix}\right] 
	\otimes \left[\begin{matrix} -1 & 0 & 1	\end{matrix}\right] =
	\left[\begin{matrix}
			-1 & 0 & 1\\
			-1 & 0 & 1\\
			-1 & 0 & 1\\
	\end{matrix}\right]	\qquad
	h_y = \left[\begin{matrix} -1 \\ 0 \\ 1\end{matrix}\right] 
	\otimes \left[\begin{matrix} 1 & 1 & 1	\end{matrix}\right] =
	\left[\begin{matrix}
			-1 & -1 & -1\\
			0 & 0 & 0\\
			1 & 1 & 1\\
	\end{matrix}\right]
\]\end{scriptsize}
~\\
Sobel-Filter:
\begin{scriptsize}\[
	h_x = \left[\begin{matrix} 1 \\ 2 \\ 1\end{matrix}\right] 
	\otimes \left[\begin{matrix} -1 & 0 & 1	\end{matrix}\right] =
	\left[\begin{matrix}
			-1 & 0 & 1\\
			-2 & 0 & 2\\
			-1 & 0 & 1\\
	\end{matrix}\right]	\qquad
	h_y = \left[\begin{matrix} -1 \\ 0 \\ 1\end{matrix}\right] 
	\otimes \left[\begin{matrix} 1 & 2 & 1	\end{matrix}\right] =
	\left[\begin{matrix}
			-1 & -2 & -1\\
			0 & 0 & 0\\
			1 & 2 & 1\\
	\end{matrix}\right]
\]\end{scriptsize}
\\\\
Lösung in MATLAB:
\lstset{language=Matlab}
\lstinputlisting[firstline=1,caption=]{../Matlab/gradient1.m}
~\\
Der Gradient kann auch als Vektor dargestellt werden mit dem Betrag $\difrac{I}{r}$ und Winkel $\alpha$:
\[
	\difrac{I}{r} = \sqrt{\left(\pdifrac{I}{x}\right)^2 + \left(\pdifrac{I}{y}\right)^2} \qquad
	\alpha = \arctan \frac{\pdifrac{I}{y}}{\pdifrac{I}{x}}
\]
\\\\
Lösung in MATLAB:
\lstset{language=Matlab}
\lstinputlisting[firstline=1,caption=]{../Matlab/gradient2.m}
~\\

\subsection{Bildschärfung}
Für die Bildschärfung wird der Laplace Operator verwendet, welcher ein linearer richtungsunabhängiger Operator ist.
Er ist definiert als die Summe der zweiten Ableitung:
\[
	\Delta I = L = \pdifrac{^2 I}{x^2} + \pdifrac{^2 I}{y^2}
\]
~\\
Die Approximation für diskrete Werte $I_{m,n}$ des Bildes (nur für $x$, $y$ analog dazu):
\[\begin{aligned}
	I_{xx} &\approx \frac{I_x(x+1,y)-I_x(x,y}{1} \\
	&\approx \frac{I(x+1,y)-I(x,y}{1} - \frac{I(x,y)-I(x-1,y}{1}\\
	\pdifrac{^2 I}{x^2} &\approx I_{x+1} - 2 \cdot I_x + I_{x-1}
\end{aligned}\]
~\\
Die dazugehörige Maske:
\[
	L = \left[ \begin{matrix}
	0 & -1 & 0\\
	-1 & 4 & -1\\
	0 & -1 & 0
	\end{matrix} \right]
\]
~\\
Für die Bildschärfung wird der Laplace Operator in Kombination mit dem Identischen Operator $I$ verwendet:
\[
	I + \beta \cdot L
\]
~\\
\begin{footnotesize}
$\beta$: Grad der Bildschärfung
\end{footnotesize}
\\\\
Lösung in MATLAB:
\lstset{language=Matlab}
\lstinputlisting[firstline=1,caption=]{../Matlab/bildschaerfung.m}
~\\
